\documentclass{article}
\usepackage[UTF8]{ctex}
\usepackage{geometry}
\usepackage{multirow}
\usepackage{natbib}
\geometry{left=3.18cm,right=3.18cm,top=2.54cm,bottom=2.54cm}
\usepackage{graphicx}
\pagestyle{plain}	
\usepackage{setspace}
\usepackage{enumerate}
\usepackage{caption2}
\usepackage{datetime} %日期
\renewcommand{\today}{\number\year 年 \number\month 月 \number\day 日}
\renewcommand{\captionlabelfont}{\small}
\renewcommand{\captionfont}{\small}
\begin{document}

\begin{figure}
    \centering
 

    \label{figupc}
\end{figure}

	\begin{center}
		\quad \\
		\quad \\
		\heiti \fontsize{45}{17} \quad \quad \quad 
		\vskip 1.5cm
		\heiti \zihao{2} 《计算科学导论》个人职业规划
	\end{center}
	\vskip 2.0cm
		
	\begin{quotation}
% 	\begin{center}
		\doublespacing
		
        \zihao{4}\par\setlength\parindent{7em}
		\quad 

		学生姓名:\underline{\qquad  郝珠琳 \qquad \qquad}

		学\hspace{0.61cm} 号:\underline{\qquad 1907010204\qquad}
		
		专业班级:\underline{\qquad 计算1902 \qquad  }
		
        学\hspace{0.61cm} 院:\underline{计算机科学与技术学院}
% 	\end{center}
		\vskip 1.5cm
		\centering
		\begin{table}[h]
            \centering 
            \zihao{4}
            \begin{tabular}{|c|c|c|c|c|c|c|c|c|}
            % 这里的rl 与表格对应可以看到,姓名是r,右对齐的;学号是l,左对齐的;若想居中,使用c关键字。
                \hline
                \multicolumn{5}{|c|}{分项评价} &\multicolumn{2}{c|}{整体评价}  & 总    分 & 评 阅 教 师\\
                \hline
                自我 & 环境 & 职业 & 实施 & 评估与 & 完整性 & 可行性 &\multirow{2}*{} &\multirow{2}*{}\\
                分析& 分析& 定位 & 方案 & 调整 & 20\% & 20\% & ~&~ \\\            
                10\% & 10\% & 15\% & 15\% & 10\% & &  &~ &~\\
                \cline{1-7} 
                & & & & & & & ~&~ \\
                & & & & & & & ~&~ \\
                \hline      
            \end{tabular}
        \end{table}
		\vskip 2cm
		\today
	\end{quotation}

\thispagestyle{empty}
\newpage
\setcounter{page}{1}
% 在这之前是封面,在这之后是正文
\section{自我分析}
	自我分析即对自己进行全方位、多角度的分析,目的是认识自己、了解自己。只有认识了自己,才能对自己的职业做出正确的选择,才能选定适合自己发展的职业生涯路线,才能对自己的职业生涯目标做出最佳抉择。\par
	自我分析包括:\par
\subsection{自然条件}
性别:女\par
年龄:17\par
身体条件:身高160cm 体重51kg \par
健康状况:良好\par
居住城市:河南南阳\par
\subsection{性格分析}
优点:有上进心;有较好的组织沟通能力;乐观开朗;善与自省\par
缺点:做事不够沉稳;缺乏经验;容易情绪化;有时过于固执和计较;\par
\par
\subsection{教育与学习经历}
小学和初中就读于河南省南阳市镇平县\par
高中就读于南阳市一中\par
本科就读于中国石油大学(华东)\par
\par
\subsection{工作与社会阅历}
高中寒暑假参与过社区打扫志愿服务\par
初中寒暑假时和爸妈一起参加过很多会展(当时家里经营珠宝),曾经自已一个人管理两个柜台(虽然一件玉器也没卖出去)\par
初高中时担任过许多班干部,有较强的组织能力\par
\par
\subsection{知识、技能与经验}
弹吉他,电子琴(初级);\par
c++(基础);\par
高一得过南阳市"小作家杯"二等奖\par
全国地理协会颁发的“地理小博士”一等奖\par

\par
\subsection{兴趣爱好与特长}
看书,慢跑,骑行;\par
弹吉他,电子琴(初级)\par
经常会写一些东西,但是都是自己看的\par
平时会自学一点绘画(喜欢搭配色彩和构造人物形象)
\par
\section{环境分析}
环境分析主要是评估周边各种环境因素对自己职业生涯发展的影响。每一个人都处在一定的环境之中,职业发展必然要受到所处环境的影响,只有充分了解和把握所处环境的现状、特点、发展变化趋势,才能做到在复杂的环境中避害趋利,使你的职业生涯规划具有实际意义。\par
环境分析包括:\par
\subsection{社会环境分析}
政治:行政体制的试验改革和创新以适应不断发展的经济形式和不断变化的国家国内环境。\par
经济:利用经济发展速度降低的时候努力的调整经济的结构,是产业结构趋于合理化,第三产业服务业在国民经济中的比重上升。\par
就业:
1.需求趋长、职位较多的行业
未来几年,计算机、通信、电子等信息类专业、生命科学、高新技术等行业人员需求大,大学生就业增长空间较大。师范类毕业生仍会供不应求,环境科学类、生命科学、应用数学、法律服务、交通运输类以及工科的仪表类、纺织类需求都会增多,外语类中的复合人才、石油、煤炭、冶金等需求都会有不同程度的增长,但有些行业的容量有限。
2.需求大体保持不变的行业
主要有机械类、材料类、外语类大语种、医学类、管理专业、经济学、财政学、统计学、价格学、国民经济计划以及金融,财经等需求不会有大的增加,比较平稳。其中有些专业如医学类毕业生需要重点移往中小城市、农村及城市基层。计算机、通信、电子等信息类专业虽然需求量较大,但人才培养规模也不断增大,两长相消,信息类专业的就业场面难再火爆。
3.需求趋降的行业
从行业上看主要是农、林、牧、渔、制造、建筑业,尤其是农林类毕业生就业仍然相对困难。但不能一概而论,以上各行业高素质专业人才仍需求兴旺。建筑类虽有相当需求,但这行业目前仍处于劳动密集型,吸纳的主流仍为劳力型人员,对专业人才的需求却不旺。
从学科类型看,文、史、哲类人才社会需求趋减。哲学、社会学、历史学、人口学、宗教学等毕业生未来几年的就业形势仍难见好。

\par
\subsection{家庭环境分析}

婚姻状况:未婚;\par
经济状况:零收入;\par
家人期望:爸爸期望我在青岛落户;
         妈妈期望我在郑州落户;
         至于我将来的工作,他们支持我的选择;\par
家族传统:好像没有。\par
\par
\subsection{职业环境分析}
行业现状及发展趋势:火爆,并且未来将是人工智能的时代。\par
职业的工作内容:信息技术,计算机硬件,芯片设计,可编程设计,软件设计,动漫设计,3D:\par
工作要求:精通算法和编程;\par
有一定的逻辑思维能力和合作交流能力;\par
有良好的心理将康状态;\par
有较好的身体素质;\par
发展前景:前景可观;\par
\par


\subsection{地域与人际环境分析}
1.工作城市的气候水土:\par
青岛是海滨丘陵城市,岸线曲折,岬湾相间。东有崂山,西有大小珠山和铁镢山,北有大泽山,中部为胶莱平原。有大沽河、北胶莱河以及沿海诸河三大水系,胶州湾、鳌山湾、灵山湾三大湾群,782千米大陆海岸线、49 个海湾和 120 个海岛。青岛大陆岸线占山东省岸线的 1/4。
\par
青岛海区港湾众多,岸线曲折,滩涂广阔,水质肥沃,是多种水生物繁衍生息的场所;胶州湾、崂山湾及丁字湾口水域营养盐含量高,补充源充足,异样菌量比大陆架区或大洋区高出数倍乃至数千倍,水中有机物含量较高。尤其是胶州湾一带泥沙底质岸段,是发展贝类、藻类养殖的优良海区。该海区的浮游生物、底栖生物、经济无脊椎动物、潮间带藻类等资源也很丰富\par
2.文化特点:\par
青岛历史悠久,文化灿烂,是中国道教的发祥地之一。 早在五六千年以前的新石器时代,这里是东夷人繁衍生息的主要地区之一,遗留了丰富多彩的北辛文化、大汶口文化、龙山文化和岳石文化等。 春秋战国时期,这里建立了山东地区第二大市镇——即墨。越王勾践在琅琊(今黄岛区境内)筑台会盟,成为一代霸主。
秦始皇统一中国后,五巡天下,三登琅琊。据记载,中国最早的一次涉洋远航——徐福东渡朝鲜、日本,便起航于琅琊。
发展前景:\par
以习近平新时代中国特色社会主义思想为指导,按照中央决策部署和省委、省政府要求,发起海洋、“双招双引”、交通基础设施建设、军民融合发展、乡村振兴、突破平度莱西、国际航运贸易金融创新中心建设、“高端制造业 + 人工智能”、推进国企改革、壮大民营经济、科技引领城建设、城市品质改善提升、国际时尚城建设、高效青岛建设、“平安青岛”建设等攻势,加快建设开放、现代、活力、时尚的国际大都市,打造山东面向世界开放发展的桥头堡。
3.人脉与人际关系:\par
人脉少;人际关系良好\par
\par
\par 

\begin{figure}[h!]
\centering

\end{figure}



\section{职业定位}
在准确地对自己和环境做出了分析之后,确定适合自己行业和有实现可能的职业发展目标。职业定位时要注意与自己的自然条件、知识背景、技能特长、性格特点、兴趣爱好是否匹配,考虑与自己所处的环境是否相适应。职业定位决定了职业发展中的行为和结果,是制定职业生涯规划的关键,应当科学合理,具有可行性。\par
职业定位包括:\par

\subsection{行业领域定位与理由}
UI/网页设计师\par
想做设计,并且用到自己的专业知识
\par
\subsection{职业岗位起点定位与理由}
初级设计师,不需要太多的工作经验,对技术要求不高,容易下手
\par
\subsection{职业目标与可行性分析}
成为一名优秀的ui设计师
\par

\begin{enumerate}[(1)]
	\item 短期目标(大学4年)\par
顺利完成学业,拿到毕业证和学位证\par
每年至少阅读24本书(尽量和所学专业有关的),并且有读书笔记\par
精通一门外语\par
保持良好的身体素质的心理素质\par	
熟练掌握photoshop和corel draw\par
课余学习美术绘画\par	
	
	
	\item 中长期目标(5-10年)
	进一家比较知名的互联网公司工作
	
	
	
	
	
\end{enumerate}



\section{实施方案}
在明确了职业定位后,要制定实现职业生涯目标的行动方案,不付诸行动,职业目标只能是一种梦想。实施方案是实现职业目标的保证,尽量考虑周全、具有可操作性。\par

\begin{enumerate}[1、]
	\item 大学有许多空余时间,可以学习或者培训设计方面的知识;学好编程。
	\item 制定计划,每天落实,不拖延。
	\item 结交有能力有担当的朋友,多和能力强的人沟通交流。
	\item 工作和家庭都是生活的一部分,多联系家人,以工作为中。
	\item 通过健身,听歌来缓解压力,有一定的经济基础可以去旅游放松身心。
\end{enumerate}
\par 

	\label{table1}

\section{评估与调整}
由于影响职业生涯规划的因素很多,且大都处于动态变化之中,因此职业生涯规划应定期评估,并根据影响因素的变化和实施结果的情况及时作出调整,这样才能保证其行之有效。\par 
\subsection{评估时间}
每学年评估一次\par
\subsection{评估内容}
从成果目标、经济目标、能力目标、职务目标等方面总结,确定哪些目标已按预期实现,哪些目标商未达到,对已实现的成果总结经验,对未完成的目标分析原因。\par
\subsection{调整原则}
考虑与自身情况的匹配性、与环境的适应性、操作实施的可行性等。\par




\end{document}
