\documentclass{article}
\usepackage[UTF8]{ctex}
\usepackage{geometry}
\usepackage{natbib}
\geometry{left=3.18cm,right=3.18cm,top=2.54cm,bottom=2.54cm}
\usepackage{graphicx}
\pagestyle{plain}	
\usepackage{setspace}
\usepackage{caption2}
\usepackage{datetime} %日期
\renewcommand{\today}{\number\year 年 \number\month 月 \number\day 日}
\renewcommand{\captionlabelfont}{\small}
\renewcommand{\captionfont}{\small}
\begin{document}

\begin{figure}
    \centering
    

    \label{figupc}
\end{figure}

	\begin{center}
		\quad \\
		\quad \\
		\heiti \fontsize{45}{17} \quad \quad \quad 
		\vskip 1.5cm
		\heiti \zihao{2} 《计算科学导论》课程总结报告
	\end{center}
	\vskip 2.0cm
		
	\begin{quotation}
% 	\begin{center}
		\doublespacing
		
        \zihao{4}\par\setlength\parindent{7em}
		\quad 

		学生姓名:\underline{\qquad  郝珠琳 \qquad \qquad}

		学\hspace{0.61cm} 号:\underline{\qquad 1907010204\qquad}
		
		专业班级:\underline{\qquad 计算1902\qquad  }
		
        学\hspace{0.61cm} 院:\underline{计算机科学与技术学院}
% 	\end{center}
		\vskip 2cm
		\centering
		\begin{table}[h]
            \centering 
            \zihao{4}
            \begin{tabular}{|c|c|c|c|c|c|c|}
            % 这里的rl 与表格对应可以看到,姓名是r,右对齐的;学号是l,左对齐的;若想居中,使用c关键字。
                \hline
                课程认识 & 问题思 考 & 格式规范  & IT工具  & Latex附加  & 总分 & 评阅教师 \\
                30\% & 30\% & 20\% & 20\% & 10\% &  &  \\
                \hline
                 & & & & & &\\
                & & & & & &\\
                \hline
            \end{tabular}
        \end{table}
		\vskip 2cm
		\today
	\end{quotation}

\thispagestyle{empty}
\newpage
\setcounter{page}{1}
% 在这之前是封面,在这之后是正文
\section{引言}
  计算机科学与技术专业的学习是一项十分艰巨的劳动,不少近年来成长起来 的青年科学家和工程师都有同感。经验告诉我们,学习计算科学其至比学习基础 数学还要困难,因为其不少理论课程在深度上不比数学课程更简单,同时学生又要面对大量实践内容的学习, 知识更新周期很短。理论与实践相结合,理论与实 践的统一是计算科学的一大特点,它决定了在学习中学生要经常不断地在严密的 逻辑思维与形象的实验操作之间转换学习方式,这对大多数人不是一件轻松的事。 何况计算科学学科发展极快,在工作中对知识组织结构的补充与更新任务犹如泰 山压顶,让人喘不过气来。难怪些计算科学 大师们感叹:“计算科学是年轻人的 科学。”这就是说,一旦你选择了计算科学作为你终生为之奋斗的专业领域,就等 于你选择了一条布满荆棘的道路,一条充满艰 辛的人生之路。一个有志 于从事计算科学研究与开发的学生,必须在大学的几年学习中打下坚实的基础,才有可能 在将来学科的高速发展中,或在计算机产品的开发和快速更新换代中有所作为。\par
 《计算科学导论》内容重在引导学生怎么从科学哲学的角度去认识和学习计算科学,也算科学包括为学习后续课程准备的布尔代数的基础知识。这些内容对学生学好计算科学,在将来顺利完成学业是有益的。在学习中,本书中带有结论性的观点、面去方法和认识学生应牢记在心。这不仅因为它们是计算科学(教育)界多年来经验“深 的积累,而且,随着同学们学习的不断深入,知识的不断积累,会进一步加深对这些观点、方法的认识,有助于大家学好计算科学。


\section{对计算科学导论这门课程的认识、体会}
首先,计算机科学导论这门课是让我们对这计算机这个学科有所了解,为我们的专业学习打好基础, 这样 我们才能更好地学习这个专业。它从如何引导我们成长为一个优秀的专业技术人员这样一个问题谈起, 从 一个更一般的认识层面上掌握如何学习一门专业知识的方式方法,解决如何认识计算机科学与技术,如何 学习计算机科学与技术的问题。这门课程从计算机这一领域出发,谈到了计算科学的基本知识,计算科学 的意义、内容和方法。其中包括计算模型与二进制,通用数字计算机系统结构与工作原理,数字逻辑与集 成电路,机器指令与汇编语言,算法、过程与程序,高级语言与程序设计,系统软件与应用软件,计算机 组织与体系结构,并行计算机、通道与并行计算,计算机网络与通信,计算机图形学与图像处理,逻辑与 人工智能到数据处理与演化计,计算机科学与技术-级学科等领域内的一些重要的基本概念。还有,它所阐述的理论和方法对于我们今后的学习起到一个指导作用。它教会我们怎样才是一个科学的思维过程,面 对所要处理和解决的问题,我们要有-套怎样的科学细想方法:一个科学的认识,一套科学的方法,一个 科学的程序。看问题要从本质出发,发现问题的根本所在,这样给有利于实际问题的解决。强调了理论知 识的重要性,这也是这门学科与其它学科的明显区别。\par
例如说对于数据结扎中算法的建立我想大家应当注意以下几点:当遇到一个算 法问题时,首先要知道自己以前有没有处理过这种问题,如果见过,那么你一般会顺利地做出来;\par
如果没见过, 那么考虑以下问题:\par1.问题所要求编写的算法属于哪种类型? (如建立数据结构,修改数据结构,遍历,\par2.继续分析问题的数学本质.根据你以前的编程经验,设想一种可能是可行的解决办法。\par3.确认你的思路可行以 后,开始编写程序.在编写代码的过程中,尽可能把各种问题考虑得详细,周密程序应该具有良好的结构,并且在关键的地方配有注释.\par4.举一个例子,然后在纸上用笔执行你的程序,进一步验证其正确性. 当遇到与 你的设想不符的情况时,分析问题产生的原因是编程方面的问题还是算法思想本身有问题.要有丰富的想象 力,就是说当一条路走不通时,不要钻牛角尖,要敢于推翻自己的想法.有了这一理论上的认识,根据其结构 特点,思考实现这一-问题的算法,之后才进行实践编程。通过这门课程的学习,使我能从多方面分析,不 但明白了该门课程的重要性,让我更加重视这数据结构这门课,还让我对怎么样去学习这门课有了更加深入的了解。

\par

\subsection{高层次抽象}
用高度抽象的理论模型来刻画计算机及计算的本质问题,其特点是层次高,系统性强,且融抽象性与科学性于一体。这种教材质量较高,但对于本科生来说,难度较大,不易掌握 。

\par



\subsection{“浓缩”+“拼盘”}
将本学科的主干课程,如操作系统、数据结构、软件工程、数库系统、计算机网络等“浓缩”起来,独立成章,然后合成一个“拼盘”。其特点是内容广而散、概念多而杂、理论深而不透,学生很难理解与掌握。
表格插入样例:\par



\section{医学影像识别}
人工智能在医学影像的应用,主要是通过图像 识别和深度学习等技术,实现机器“看片”和“读片” 的功能。具体应用包括计算机辅助诊断、影像组 学、影像基因组学等。

\begin{itemize}
    \item 3.1计算机辅助诊断
    计算机辅助诊断系统( Computer Aided Diagno- sis ,CAD) ,是影像学AI应用的重要内容,它是将图 像处理.计算机视觉、医学图像分析等有效结合,通 过系统处理后对异常征象进行标注,以帮助医生快 速发现病灶,提高诊断的效率和准确率。CAD的研 究,最早可追溯到20世纪60年代,但受技术水平的 限制,研究进展较为缓慢。20 世纪90年代,随 着计算机技术.数学算法及统计学的发展,基于机 器学习和图像处理技术的CAD在医学影像诊断领域获得了快速发展,针对不同疾病的CAD研究大量 涌现"。2012年以后,由于深度卷积神经网络的 兴起、大量数据的累积和基于图像处理器的计算能 力大幅提升,深度学习在医疗图像上的研究空前高 涨,从而使CAD的架构更为简化,诊断更为. 精确。
    目前,CAD可应用于多种影像技术对疾病的检 测和诊断,在肺结节和肺癌筛查、乳腺癌筛查和前 列腺癌影像诊断中应用较广,且表现较为突出“。一些CAD系统的性能已经接近甚至超过放射科医生。
    
    \item 3.2影像组学
    
    影像组学( radiomics)源自于CAD,于2012年由荷兰学者lambin等旧首次提出。作为-种大数据 图像分析方法,影像组学是从医学影像中高通量地 提取海量数据,并对数据信息进行深层次的挖掘、. 分析和解读,以发现隐含在图像中的额外信息,最. 高效地利用影像学检查结果,实现临床辅助决 策"。其基本分析流程包括五个环节:①图像采 集:主要通过CT、MRI. PET等影像扫描方式采集图 像;②图像分割:将影像中的异常组织(如肿瘤)或. 特定解剖组织(如海马)等分割为1个或多个感兴 趣区域;③特征提取:对感兴趣区域进行影像特征 提取,主要包括强度、形状、纹理、位置等特征;④量 化分析:对上述特征进行统计分析,常用的分析方法 有重复测量信度分析、主成分分析、相关性分析和随 机森林等;⑤模型构建:通过机器学习(深度学习)方 法建立基于影像组学特征的预测和分类模型1。
    影像组学突破了基于形态学及半定量分析的 传统影像医学模式,可提供以往基因检测或病理检 查才能提供的信息,对于临床医生从早期图像中获取 诊断信息有着重要帮助。目前,国内外影像组学主要 集中于肿瘤(如肺癌、乳腺癌、胶质母细胞瘤、肝癌等) 的相关研究,包括肿瘤的定性诊断、分级分期、基因表 型预测、治疗方法选择及疗效预后评估等明。
    
    \item 3.3影像基因组学
    20世纪90年代初的基因组革命,推动着医学 研究从基因水平探究疾病的基础机制,以实现精准医疗。传统的基因分析手段依赖于有创的活检取 材或术后病理组织来进行,具有一定的风险和潜在 的并发症。相比之下,医学影像具有非侵入性、高 分辨率、时空连续性等特点,在展现复杂疾病表型 差异的过程中具有独特的优势。基于此,2000年以 来陆续有学者将基因组数据和影像数据关联起来 进行分析、挖掘,由此形成了新的研究方向,即影像 基因组学( radiogen omics) 6D-2 。影像基因组学与 影像组学的细微区别在于,它不仅从影像数据(包 括CT.MRI.PET等)中提取能反应个体健康状态的 定量影像表型特征,还要从生物组学数据(包括基 因组、转录组学和表观组学等)中提取基因型特征, 并通过人工智能技术完成基因型特征与定量表型 特征的关联与融合分析,从而更好地实现对疾病的 非侵入式诊断预后预测和疗效评估,是目前生物 医学最有前景的研究领域之一。
    
    \item 3.4问题与挑战\par
    1机器性能问题
    目前,Al医疗的研究和开发在我国还处于起步 阶段。尽管不少研究或产品已在实验室取得了骄 人的成绩,但由于大多数产品都是基于公开数据集 训练而来,不能反映真实的、复杂的临床环境,- -旦 落地临床应用,难以保持测试数据的高准确率,错 标、漏标、多标现象时有发生,需要临床医生花费大 量时间精力进行标注和复查。同时,由于AI技术尚 处于发展阶段,某些技术尚未完全成熟,导致机器 性能还不够稳定,同一Al模型应用于不同地域的医 院时,可能会出现数据差异,需要进行精细微调。 另外,目前AI影像产品在单病种领域进展迅速,如 在肺结节筛查、糖尿病、眼病、儿童骨龄检测等诸多 细分领域取得了显著成绩,但在复杂的临床使用环 境中依然面临较大挑战。例如,肺结节筛查只是胸 部CT检查的一小部分需求,大量的肺炎、肺结核、 慢阻肺等疾病所造成的“同病异影、异病同影”现象 依然难以检出,使得AI的应用范围非常局限”。这 些都在- -定程度 上影响了临床医师的应用积极性。\par
    2隐私泄 露问题
    人工智能在医学影像应用中,需要采集和挖掘 患者的大量信息,包括患者的基本信息.健康状况、 疾病状况、生物基因信息等,一旦泄露后果不堪设 想。如保险公司在掌握个人病史的情况下,可能提 高保险费用;用人单位可能把个人健康档案作为是 否聘用的重要依据等叫。患者隐私泄露的风险主 要来自于两个方面: -是掌握数据的个人或机构主 动泄露,如2016年,英国伦敦皇家自由医院将大约160万名患者的信息交给DeepMind公司进行医学 研究,因未能充分保护患者隐私和数据来源的正当 性受到质疑,被英国信息委员会勒令整改叫。二是 被他人非法窃取。
    近年来,影像基因组学在肿瘤和精神疾病等复 杂疾病的研究领域不断发展,在脑肿瘤.肺癌、乳腺 癌等方面均有所探索。当然,影像基因组学的数据 分析和判断,仍需要有经验的放射科医师或专家才 能完成,人的智力是主导成分,而计算机则帮助医 师计算和分析,提供有价值的信息。随着研究的进 一步深入,影像基因组学将在医学领域尤其是癌症 研究工作中发挥更加积极的作用,并很有可能改变 癌症患者的诊断、治疗和预后。\par
    3数据质量问题
    影像数据的质量决定了人工智能模型学习的 结果,标准的影像数据和规范的数据标注是医疗影 像AI发展的关键。然而,尽管当前我国医疗机构积 累了大量的影像数据,但由于缺乏统一的标准和规 范,并未实现影像图像质量和格式的同质化。不同 的医疗机构由于信息化建设水平不一,不同厂商、 不同档次的影像设备存在图像质量、算法重建和参 数设置的差异,即使同一台设备,造影剂剂量.扫描 层厚、成像深度和增益等也会对图像产生影响,导 致影像数据标准各异,图片质量参差不齐。同时, 影像数据必须经过临床经验丰富的医生标注才能 用于机器学习,但数据标注需要耗费大量时间和精 力,高质量的、标注过的数据资源相对有限,加上医 院之间的数据共享和互通程度较低,真正能够接触 并利用到大规模优质医疗数据的开发者寥寥无几。 相当-部分AI企业用于训练的数据只能来自有限 的公开数据集或自备数据库,存在着数据量过小、. 影像质量较低,标注不规范甚至标注错误等问题, 势必会影响机器学习的准确性和普适性。因此,发 展医学影像AI,图像数据亟须规范化和标准化。
    \par
    4算法偏见问题
    当前在医学影像中应用最多的深度学习算法, 使用了大规模的神经网络,包含了更多的计算隐 层,具备强大的自我学习和自我编程能力,其复杂 性和不确定性使得人工智能存在难以捉摸的“黑盒 子”,即使是开发者本人,也很难解释它的内部运作 方式和某个具体行动背后的逻辑2。这种不透明 性和不可解释性,使得某些算法偏见难以被觉察。 同时,当前深度学习算法并未实现真正意义上的“智能”,它只不过是基于高速运算能力和规模数据.\par
    
\end{itemize}


\section{总结}
总的来说,通过这门课的学习,我了解的我所学习的这个学科。也知道了我该走的方向。正如书中前言部 分中说到的:本书的内容生在引导学生怎么从科学哲学的角度去认识和学习计算科学。这些内容对学生学 好计算科学,顺利完成学业是有益的。我也学习到了很多东西,明白了我现在所学有关其它计算机方面课 程的重要性。所以,我一定会认真地学习好相关的课程。有人说过: 一旦你选择了科学作为你终生为之奋 斗的专业领域,就等于你选择了-条布满荆刺的路,一条充满艰辛的人生之路,一个有志于从事于计算科 学研究与开发的学生,发须在大学的几年学习中打下坚实的基础,才有可能在将来学科的高速发展中,或 在计算机产品开发和快速更新换代中有所作为。正因为这样,我要更加努力地学习相关方面的知识,学好 基础课程,丰富自己的头脑。在以后相关专业的学习过程中,我将一直受益于这一门课所教会我的科学的 认识与学习方法。不管是对于哪一学科的问题,我都会先去认识事物的本质,发现问题的根本,深刻思考 过后,在去从实际中解决它。特别是自己碰到的计算机问题,努力在这一 领域中获得巨大的成就。
\par


\section{附录}


\begin{figure}[tph]
	\centering
	\includegraphics[width=0.4\linewidth]{"C:/Users/86176/Pictures/Camera Roll/z"}
	\caption{}
	\label{fig:z}
	
\end{figure}
\begin{figure}[tph]
	\centering
	\includegraphics[width=0.4\linewidth]{"C:/Users/86176/Pictures/Camera Roll/x"}
	\caption{}
	\label{fig:x}
\end{figure}
\begin{figure}[tph]
	\centering
	\includegraphics[width=0.4\linewidth]{"C:/Users/86176/Pictures/Camera Roll/v"}
	\caption{}
	\label{fig:v}
\end{figure}
\begin{figure}[tph]
	\centering
	\includegraphics[width=0.4\linewidth]{"C:/Users/86176/Pictures/Camera Roll/n"}
	\caption{}
	\label{fig:n}
\end{figure}
\begin{figure}[tph]
	\centering
	\includegraphics[width=0.4\linewidth]{"C:/Users/86176/Pictures/Camera Roll/m"}
	\caption{}
	\label{fig:m}
\end{figure}
\begin{figure}[tph]
	\centering
	\includegraphics[width=0.4\linewidth]{"C:/Users/86176/Pictures/Camera Roll/c"}
	\caption{}
	\label{fig:c}
\end{figure}
\begin{figure}[tph]
	\centering
	\includegraphics[width=0.4\linewidth]{"C:/Users/86176/Pictures/Camera Roll/b"}
	\caption{}
	\label{fig:b}
\end{figure}



\hspace*{\fill} \\



https://github.com/HAOZHULIN?tab=repositories





\par
 腾讯研究院,中国信通院互联网法律研究中心
Chinese Medieal Ethics	Vol. 32 No. 08\par


人工智能[M]. 北京中国人民大学出版社
 国家卫生健康委员会.2010年我国卫生事业发展统计公报
 http: //www. nhfpe. gov. cn/ mohwsbw-
stjxxzx/s7967 /201 104/51512. shtml.\par


国家卫生健康委员会. 2017年我国卫生健康事
业发展统计公报DEB/0L]. (2018 -06- 12)
2019 -02 - 15] . htp: /www. nhfpe. gov. en/ .
guihuaxxs/s10743 /201806/44e3edfel 1fa4c7f 928e
879d435b6a18. shtm!? from = singlemessage\&isa ppinstalled=1\par


金征字.人工智能医学影像应用:现实与挑战
放射学实践, 萧毅,刘士远.\par


人工智能将改变影像医学的未来\par

科技与金融,2018(10):11-15.\par

 国家卫生计生委办公厅关于《职业病防治法》等
法律法规落实情况监督检查工作的通报\par
[EB/0I].(2016-12-07) E019-02 - 16].
 htp://www. moh. gov. cn/zhjej/s35 77/201612 /923d6240ad44h58ad1 88da6 14c8fd38. shtml.\par
 
 
胡冉,李莹莹 人工智能+影像应用深度研究报告发布 (2018 -05 
 http: /www. sobu. conm/a/233505635. 324186.8]\par
 
 
  王云.人工智能加码医疗影像上海信息化\par
  
  
 国务院.新-代人工智能发展规划团.2017-07 -08\par
 
 
健康点.2018医疗Al火出新高度!\par


 BA与CPS争相布局EB/OL.(2018-07-10) 2019-02- 15].\par
 http: //www. sohu. com/a/240304399\par
 
 
高歌,马帅,王霄英.计算机辅助诊断在医学影像诊断中的基本原理和应用进展放射学实践,2016,31(12):1127 -1129.\par



 Zhang W, Li R, Deng H,et al. Deep convolutional
neural networks for multi - modality isointense in- fant brain image segementation D] . Neurolmage ,
2015, 108:214 -224.\par



 Liang M, Tang W, Xu DM,et al. Low - dose CT
screening for lung cancer: computer - aided detee-
tion of missed lung cancers Radiolongy \par

14 Patel TA,Puppala M,Ogunti RO,et al. Crelaing
mammographie and pathologie findings in elinical
decision support using natural language processing
and data mining methods [] . Cancer, 2017, 123\par


檀韬,喻秉斌, 吴山东.医学影像诊断及介入式
手术的人工智能应用\par



 Lambin P, Rios - Velazquer E,Lejenar R,et al.
Radiomics: Extracting more information from medi-
cal images using advanced feature analysis 0] . Eur
J Cancer ,2012 ,48(4):441 -446.\par

谭俊,袁少勋,明文龙,等 影像基因组学分析方法\par

魏强,陆平.人工智能算法面临伦理困境\par


陈永晔,张恩龙,张家慧,等基于影像学的多种人工智能算法在肿瘤研究中的应用进展\par


包桉冰,徐佩. 医疗人工智能的伦理风险及应对策略\par



\bibliographystyle{plain}
\bibliography{references}



\end{document}
